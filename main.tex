\documentclass{article}

% Language setting
% Replace `english' with e.g. `spanish' to change the document language
\usepackage[danish]{babel}
% Set page size and margins
% Replace `letterpaper' with `a4paper' for UK/EU standard size
\usepackage[a4paper,top=2cm,bottom=2cm,left=3cm,right=3cm,marginparwidth=1.75cm]{geometry}

% Useful packages
\usepackage{amsmath}
\usepackage{graphicx}
\usepackage[colorlinks=true, allcolors=blue]{hyperref}
\usepackage{array}
\usepackage{subfig}
\usepackage{subfigure}
\usepackage{titlesec}

\title{CDIO delopgave 0}
\author{Jakob Agergaard}


\begin{document}

\titlespacing{\section}
    {0pt}{2em}{1em}




\begin{titlepage}
\begin{center}

    \includegraphics[width=0.25\textwidth]{Billeder/DTULogo.png} \\
    \vspace{0.5cm}
    \Large
    \textbf{02314\hspace{1cm}62531\hspace{1cm}62532} \\
    Indledende programmering, Udviklingsmetoder til IT-systemer og Versionsstyring og testmetoder
    \vspace{0.4cm}
    \hrule
    
    \vspace*{0.5cm}
    \huge
    \textbf{CDIO delopgave 0}\\
    \LARGE
    Gruppe 17
    \vspace{0.5cm}
    \hrule
    \vspace{0.2cm}

    \large
    \begin{tabular}{m{10em} m{8em} m{8em} m{10em}}
    Jakob Skov Agergaard\vfill s224570 & \includegraphics[width=0.2\textwidth]{Billeder/JakobFoto.png} & \includegraphics[width=0.2\textwidth]{Billeder/PhilipFoto.jpg} & Philip Muff Førrisdahl\vfill s224566 \\
    Mads Fogelberg Hansen\vfill s224563 & \includegraphics[width=0.2\textwidth]{Billeder/FotoMads.jpg} & \includegraphics[width=0.2\textwidth]{Billeder/EsbenFoto.png} & Esben Skovmand Elnegaard \vfill s224555  \\
    Jarl Boyd Roest\vfill s224556 & \includegraphics[width=0.2\textwidth]{Billeder/JarlFoto.png}
    \end{tabular}

    \vfill
    
    
    \vspace{1cm}
    \LARGE
    18. september 2022

    \vspace{1cm}
    
\end{center}
\end{titlepage}


\normalsize
\begin{abstract}
     Her kommer der et resume af opgaven
\end{abstract}

\tableofcontents

\section{Timeregnskab}

\section{Indledning}

Her kommer indledningen...

\section{Projekt-planlægning}
\begin{itemize}
    \item [1/11] Her laver vi en smule opgaveprioritering og udarbejder en hurtig kravsliste over funktionelle krav.\\
    Evaluering på dagen: Alt nået
    \item [2/11] Lave en primær use case beskrivelse (brief) og identificere andre sub-use cases. Udarbejde et use case diagram og starte på domænemodel.\\
    Evaluering på dagen: Alt nået, samt domænemodel færdiggjort og start på designklasse diagrammet. Vi har implementeret de første og simpleste klasser i IntelliJ.
    \item [4/11] Denne dag skal der laves system sekvensdiagram samt sekvensdiagram.
    
\end{itemize}
efter vi har lavet vores design klasse diagram vil starte med at lave den klasse der har færrest bindinger til de andre 


\section{Krav/Analyse}
\subsection{Kravsliste | funktionelle krav}
\begin{itemize}
    \item Skal kunne spilles af 2-4 spillere
    \item Spilleplade med 40 felter
    \item Man skal kunne købe og derefter eje felter
    \item Spillerne skal have hver deres pengebeholdning
    \item Hvis én pengebeholdning går i minus slutter spillet
    \item to terninger med seks sider
    \item En brik per spiller, som kan rykke
    \item 
\end{itemize}

\subsection{Use case beskrivelse }
\textbf{Scope:} Spil Monopoly Junoir  
\\
\textbf{Level:}
\\
\textbf{Primary actor:} Spillere 1 - 4 
\\
\textbf{Stakeholders and interests:} 
\item Spillerne vil spille et underholdende spil med formål at vinde. 
\item Projektvejlederen vil have et Monopoly Junoir spil som vil tilfredsstille kundens vision og krav
\item Kunden vil have et monopoly spil, som har taget udgangspunkt i den fremlagte vision og de dertil hørende Monopoly Junoir regler 
\\
\textbf{Preconditions:} Computeren, som skal køre Monopoly Junoir skal have den nyeste version af Java. 
\\
\textbf{Succes guarantee:} Når en spiller går fallit tæller man sammen og finder en vinder 

\textbf{Sub use cases (beskrevet på 'brief format'):}
\begin{itemize}
    \item Ryk brik
    \item Vælg spillere
    \item Slå terninger
    \item Ændre pengebeholdning
    \item Skift tur
    \item Køb felt
\end{itemize}

\textbf{Use case skrevet på 'fully dressed' format:}\\


\textbf{Main Scenario}
\begin{enumerate}
\itemsep-0.5em
    \item Første spiller slår med terningerne
    \item Systemet lægger øjnene på terningerne sammen
    \item Første spillers brik bliver nu rykket det antal felter som terningernes øjne viser
    \item Spilleren lander på en grund der ikke er købt endnu. System fortæller spilleren hvilket felt de er landet på, og spilleren betaler samtidig det beløb som feltet koster.
    \item Pengene bliver trukket fra spillerens pengebeholdning
    \item Systemet skifter tur til næste spiller.\\
    \textit{Punkterne 1-6 bliver nu gentaget for næste spiller}
\end{enumerate}
\textbf{Main Succes Scenario}
\begin{enumerate}
\itemsep-0.5em
    \item Spilleren har slået med terningerne, rykket sine brikker og evt. købt/betalt/fået beløb, og turen er gået videre til næste spiller. 
\end{enumerate}
\textbf{Extension (Alternate Succes Scenario): }
\begin{enumerate}
    \item [4.a] Spilleren lander p̊a en grund der ikke er købt endnu. Systemet fortæller spilleren hvilket felt de er landet på. Spilleren har dog ikke nok penge til at betale feltets pris.
    \subitem 1. Spillet slutter da en spiller er gået fallit. Alle spillernes pengebeholdning bliver talt op, og den med flest penge vinder.
    \item [4.b] Spilleren lander på en grund der er købt. System fortæller spilleren hvilket felt de er landet på, og spilleren betaler samtidig husleje til spilleren som ejer feltet.
    \item [4.c] Spilleren lander på et felt uden handling.
\end{enumerate}

\section{Design}

\section{Implementering}

\section{Test}

\section{Konklusion}

\section{Bilag}
\subsection{Litteratur}
\subsection{Kode}

\end{document}